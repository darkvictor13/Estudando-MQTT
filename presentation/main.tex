\documentclass[12pt]{beamer}
\usepackage[utf8]{inputenc}
\usepackage[portuguese]{babel}
\usepackage{graphicx}
\usepackage{colortbl}
\usepackage{color}
\usepackage{breqn}
\usepackage{listings}

\graphicspath{{./images/}}
\setbeamertemplate{caption}[numbered]

\definecolor{dkgreen}{rgb}{0,0.6,0}
\definecolor{gray}{rgb}{0.5,0.5,0.5}
\definecolor{mauve}{rgb}{0.58,0,0.82}
\definecolor{laranja_claro}{rgb}{1,0.9,0.5}
\definecolor{laranja_escuro}{rgb}{1,0.5,0.2}
\definecolor{azul_claro}{rgb}{0.5,0.9,1}

\lstset{frame=tb,
    language=C,
    frame=tb,
    aboveskip=3mm,
    belowskip=3mm,
    showstringspaces=false,
    columns=flexible,
    basicstyle={\small\ttfamily},
    numbers=left,
    numberstyle=\tiny\color{gray},
    keywordstyle=\color{blue},
    commentstyle=\color{dkgreen},
    stringstyle=\color{mauve},
    breaklines=true,
    breakatwhitespace=true,
    xleftmargin=.05\textwidth,
    xrightmargin=.05\textwidth,
    tabsize=4,
}

\definecolor{azul}{rgb}{0,0,.5}
\setbeamertemplate{navigation symbols}{}

\usetheme{Frankfurt}
\usecolortheme[named=azul]{structure}

%% Definindo o Autor e o título
\newcommand{\prof}{Renato Bobsin Machado}
\newcommand{\materia}{Redes de Computadores}

\author[Grupo: MQTT]{Larissa L. Wong \and Marco A. G. Pedroso \and Victor E. Almeida}

\title{Colocar Titulo}
\subtitle{Demonstrando o software}
\date{\today}
\institute{UNIOESTE}

\begin{document}
\frame{\titlepage}

\begin{frame}
\frametitle{Conteúdo}
\tableofcontents
\end{frame}

\section{Introdução}\label{Introdução}
\begin{frame}
    \frametitle{Mão na massa!!}
    \begin{figure}
        \centering
        \includegraphics[width=.3\textwidth]{pizza.png}
        \includegraphics[width=.3\textwidth]{rolo.png}
        \includegraphics[width=.3\textwidth]{burrito.png}
    \end{figure}
\end{frame}

\section{Conclusão}
\begin{frame}
    \frametitle{Agradecimentos}
    \centering
    \Huge{Perguntas?}
    \begin{figure}
        \centering
        \includegraphics[width=.3\textwidth]{alerta.png}
        \includegraphics[width=.3\textwidth]{perigo.png}
        \includegraphics[width=.3\textwidth]{eletricidade.png}
    \end{figure}
    \Huge{Obrigado pela atenção}
\end{frame}

\end{document}
