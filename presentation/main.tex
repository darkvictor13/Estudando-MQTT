\documentclass[12pt]{beamer}
\usepackage[utf8]{inputenc}
\usepackage[portuguese]{babel}
\usepackage{graphicx}
\usepackage{colortbl}
\usepackage{color}
\usepackage{breqn}
\usepackage{listings}
\usepackage{hyperref}
\usepackage{beamerthemeshadow}
\usepackage{multirow}

\graphicspath{{./images/} {../diagrams/} {../doc/images}}
\setbeamertemplate{caption}[numbered]

\definecolor{dkgreen}{rgb}{0,0.6,0}
\definecolor{gray}{rgb}{0.5,0.5,0.5}
\definecolor{mauve}{rgb}{0.58,0,0.82}
\definecolor{laranja_claro}{rgb}{1,0.9,0.5}
\definecolor{laranja_escuro}{rgb}{1,0.5,0.2}
\definecolor{azul_claro}{rgb}{0.5,0.9,1}

\lstset{frame=tb,
    language=C,
    frame=tb,
    aboveskip=3mm,
    belowskip=3mm,
    showstringspaces=false,
    columns=flexible,
    basicstyle={\small\ttfamily},
    numbers=left,
    numberstyle=\tiny\color{gray},
    keywordstyle=\color{blue},
    commentstyle=\color{dkgreen},
    stringstyle=\color{mauve},
    breaklines=true,
    breakatwhitespace=true,
    xleftmargin=.05\textwidth,
    xrightmargin=.05\textwidth,
    tabsize=4,
}

\definecolor{azul}{rgb}{0,0,.5}
\setbeamertemplate{navigation symbols}{}

\usetheme{Frankfurt}
\usecolortheme[named=azul]{structure}

\addtobeamertemplate{navigation symbols}{}{%
    \usebeamerfont{footline}%
    \usebeamercolor[black]{footline}%
    \hspace{1em}%
    Página~\insertframenumber~de~\inserttotalframenumber
}

%% Definindo o Autor e o título
\newcommand{\prof}{Renato Bobsin Machado}
\newcommand{\materia}{Redes de Computadores}

\author[Grupo: MQTT]{Larissa L. Wong \and Marco A. G. Pedroso \and\\Victor E. Almeida}

\title{Protocolo MQTT para sistemas de IoT}
\subtitle{Um estudo técnico/prático}
\date{\today}
\institute{UNIOESTE}
\logo{\includegraphics[height=1cm]{logo_unioeste.jpg}}

\begin{document}
\frame{\titlepage}

\section{Introdução}\label{Introdução}

\begin{frame}
\frametitle{Conteúdo}
\tableofcontents
\end{frame}

\begin{frame}
    \frametitle{Colocar mais algo para introduzir}
    TODO
\end{frame}

\section{História}\label{História}
\begin{frame}
    \frametitle{Criação do Protocolo}
    Começou a ser projetado durante a década de 1990 por:
    \begin{columns}[c]
        \begin{column}{.5\textwidth}
            \begin{figure}[!htb]
                \centering
                \includegraphics[width=.75\textwidth]{andy_stanford_clark1}
                \caption{\label{fig:andy}Andy Stanford-Clark da IBM}
            \end{figure}
        \end{column}
        \begin{column}{.5\textwidth}
            \begin{figure}[!htb]
                \centering
                \includegraphics[width=.77\textwidth]{arlen_nipper1}
                \caption{\label{fig:arlen}Arlen Nipper da Cirrus Link/Eurotech}
            \end{figure}
        \end{column}
    \end{columns}
\end{frame}

\begin{frame}
    \frametitle{Problemas a serem resolvidos}
    \begin{itemize}
        \item Resolver o problema de conexão de oleodutos via satélite.
        \item Limitações:
            \begin{itemize}
                \item Alta latência;
                \item Baixa largura de banda;
                \item Dispositivos com pouca bateria.
            \end{itemize}
    \end{itemize}
        
\end{frame}

\begin{frame}
    \frametitle{Requisitos do Protocolo}
    \begin{itemize}
        \item Implementação simples;
        \item Uso de QoS, \textit{Quality of Service} por quem publica a mensagem;
        \item Uso eficiente de largura de banda, baixo \textit{overhead};
        \item Baixo custo energético para envio;
        \item Possibilidade de enviar qualquer tipo de dado;
        \item Possibilidade de manter conexões ativas, prontas para enviar e receber dados;
    \end{itemize}
\end{frame}

\begin{frame}
    \frametitle{Fase do protocolo proprietário}
    \begin{figure}[!htb]
        \centering
        \includegraphics[width=\textwidth]{logo_mqtt}
        %\caption{\label{fig:logo_mqtt}Logo do MQTT}
    \end{figure}
    \begin{itemize}
        \item Primeira versão implementada no ano de 1999;
        \item Batizado MQTT, \textit{MQ Telemetry Transport}, em referência ao produto da IBM MQ Series
        \item Muito utilizado embarcado em produtos da IBM.
    \end{itemize}
\end{frame}

\begin{frame}[allowframebreaks]
    \frametitle{Fase do protocolo aberto}
    \begin{itemize}
        \item Demanda/Aplicabilidade IoT;
        \item Em 2010 o protocolo se tornou livre;
        \item Primeira versão lançada 3.1;
        \item Investimentos da IBM através da Eclipse Foundation para criar um ecossistema em torno do protocolo.
    \end{itemize}
    \begin{figure}[!htb]
        \centering
        \includegraphics[width=.5\textwidth]{eclipse_logo}
        %\caption{\label{fig:eclipse_logo}Logo da Eclipse Foundation}
    \end{figure}
    \framebreak
    \begin{figure}[!htb]
        \begin{columns}[c]
            \begin{column}{.5\textwidth}
                \includegraphics[width=.9\textwidth]{paho_logo}
            \end{column}
            \begin{column}{.5\textwidth}
                \includegraphics[width=.9\textwidth]{Mosquitto_logo}
            \end{column}
        \end{columns}
        \caption{\label{fig:logos_ferramentas_mqtt}Exemplos de aplicações do ecossistema MQTT}
    \end{figure}
    \framebreak
    \begin{itemize}
        \item No ano de 2013 a IBM buscou padronização com a OASIS;
        \item 29 de outubro de 2014 o MQTT foi aprovado como padrão pela OASIS na sua versão 3.1.1
    \end{itemize}
    \begin{figure}[!htb]
        \centering
        \includegraphics[width=.8\textwidth]{Oasis_logo}
        %\caption{\label{fig:Oasis_logo}}
    \end{figure}
\end{frame}

\begin{frame}
    \frametitle{Fase atual do protocolo}
    \begin{itemize}
        \item A última versão 5.0 março de 2019;
        \item Funcionalidades modernas como:
            \begin{itemize}
                \item facilidade de conexão e interação com a nuvem;
                \item Tratamento de erros;
            \end{itemize}
        \item Implementações de clientes para diversos sistemas e linguagens;
        \item Implementação de diversos brokers;
        \item Utilizado por grandes empresas tanto software aberto quanto proprietário.
    \end{itemize}
\end{frame}

\section{Aplicações}\label{Aplicacoes}

\section{Embasamento teórico}\label{Embasamento teórico}

\begin{frame}
    \frametitle{Titulo}

    \begin{table}[h!]\caption{Estrutura básica comum a todos os pacotes MQTT.}
        \centering
        \begin{tabular}{|l|}
            \hline
            \textbf{Cabeçalho fixo}     \\ \hline
            \textbf{Cabeçalho variável} \\ \hline
            \textbf{Carga útil}         \\ \hline
        \end{tabular}
    \end{table}
\end{frame}

\begin{frame}
    \frametitle{Titulo}
    \begin{table}[h!]\caption{Estrutura do cabeçalho fixo utilizado no protocolo MQTT.}
        \centering
        \resizebox{\textwidth}{!}{%
            \begin{tabular}{|c|cccccccc|}
                \hline
                \textbf{Bit} &
                \multicolumn{1}{m{.08\textwidth}|}{\textbf{7}} &
                \multicolumn{1}{m{.08\textwidth}|}{\textbf{6}} &
                \multicolumn{1}{m{.08\textwidth}|}{\textbf{5}} &
                \multicolumn{1}{m{.08\textwidth}|}{\textbf{4}} &
                \multicolumn{1}{m{.08\textwidth}|}{\textbf{3}} &
                \multicolumn{1}{m{.08\textwidth}|}{\textbf{2}} &
                \multicolumn{1}{m{.08\textwidth}|}{\textbf{1}} &
                \multicolumn{1}{m{.08\textwidth}|}{\textbf{0}} \\ \hline
                %\textbf{0} \\ \hline
                \textbf{Byte 1} & \multicolumn{4}{c|}{\textbf{Tipo de pacote MQTT}} & \multicolumn{4}{c|}{\textbf{Sinais específicos do pacote}} \\ \hline
                \textbf{Byte 2} & \multicolumn{8}{c|}{\multirow{2}{*}{\textbf{Espaço restante}}}                                                 \\ \cline{1-1}
                \textbf{. . .}  & \multicolumn{8}{c|}{}                                                                                          \\ \hline
            \end{tabular}%
        }
    \end{table}
\end{frame}

\begin{frame}
    \frametitle{Titulo}
    \begin{table}[]\caption{Tipos de pacotes utilizados pelo protocolo MQTT -- Parte I.}
        \centering
        \resizebox{\textwidth}{!}{%
            \begin{tabular}{|l|l|l|l|}
                \hline
                \multicolumn{1}{|c|}{Tipo} & \multicolumn{1}{c|}{Código} & \multicolumn{1}{c|}{Remetente} & \multicolumn{1}{c|}{Descrição} \\ \hline
                Reserved    & 0  & -        & Reservado                                \\ \hline
                CONNECT     & 1  & Cliente  & Solicitação 				de conexão     \\ \hline
                CONNACK     & 2  & Servidor & Confirmação de conexão                   \\ \hline
                PUBLISH     & 3  & Ambos    & Publicar mensagem                        \\ \hline
                PUBACK      & 4  & Ambos    & Publicar confirmação (QoS 1)             \\ \hline
                PUBREC      & 5  & Ambos    & Publicar recebimento (QoS 2 – parte 1)   \\ \hline
                PUBREL      & 6  & Ambos    & Publicar lançamento (QoS 2 – parte 2)    \\ \hline
                PUBCOMP     & 7  & Ambos    & Publicar conclusão (QoS 2 – parte 3)     \\ \hline
            \end{tabular}
        }
    \end{table}
\end{frame}

\begin{frame}
    \frametitle{Titulo}
    \begin{table}[]\caption{Tipos de pacotes utilizados pelo protocolo MQTT -- Parte II.}
        \centering
        \resizebox{\textwidth}{!}{%
            \begin{tabular}{|l|l|l|l|}
                \hline
                \multicolumn{1}{|c|}{Tipo} & \multicolumn{1}{c|}{Código} & \multicolumn{1}{c|}{Remetente} & \multicolumn{1}{c|}{Descrição} \\ \hline
                SUBSCRIBE   & 8  & Cliente  & Solicitação de inscrição                 \\ \hline
                SUBACK      & 9  & Servidor & Confirmação de inscrição                 \\ \hline
                UNSUBSCRIBE & 10 & Cliente  & Solicitação de cancelamento de inscrição \\ \hline
                UNSUBACK    & 11 & Servidor & Confirmação de cancelamento de inscrição \\ \hline
                PINGREQ     & 12 & Cliente  & Solicitação de PING                      \\ \hline
                PINGRESP    & 13 & Servidor & Resposta de PING                         \\ \hline
                DISCONNECT  & 14 & Ambos    & Notificação de desconexão                \\ \hline
                AUTH        & 15 & Ambos    & Troca de autenticação                    \\ \hline
            \end{tabular}
        }
    \end{table}
\end{frame}

\begin{frame}[allowframebreaks]
    \frametitle{Titulo}
    \begin{itemize}
        \item \textbf{CONNECT} – é o primeiro pacote que deve ser enviado pelo cliente no momento após ter estabelecido um conexão de rede com servidor. Entre as informações enviadas se encontra um identificador único para o cliente, entre outras informações opcionais como usuário e senha.
        \item \textbf{CONNACK} – é o pacote enviado pelo servidor em resposta ao recebimento do pacote CONNECT enviado pelo cliente, e que deve ser recebido antes do pacote AUTH. Em caso do cliente não receber um pacote do tipo CONNACK em um tempo razoável o mesmo deve encerar a conexão com o servidor e tentar novamente.
        \item \textbf{PUBLISH} – são os pacotes encarregados de carregar as mensagens entre o servidor e os clientes podendo ir em ambas as direções.
        \item \textbf{PUBACK} – é o pacote enviado em resposta a um PUBLISH realizado com QoS 1.
        \item \textbf{PUBREC} – é o pacote enviado em resposta a um PUBLISH representa a segunda parte do intercambio de mensagens realizado com QoS 2.
        \item \textbf{PUBREL} – é o pacote enviado em resposta a um PUBREC e representa a terceira parte do intercambio de mensagens realizado com QoS 2.
        \item \textbf{PUBCOMP} – é o pacote enviado em resposta a um PUBREL e representa a quarta e última parte do intercambio de mensagens realizado com QoS 2.
        \item \textbf{SUBSCRIBE} – é o pacote enviado pelo cliente ao servidor como solicitude para a criação de uma ou mais assinaturas, registrando o interesse em receber informação de um ou mais tópicos, junto com o tipo de qualidade do serviço máxima (QoS) desejada com a qual o servidor pode enviar uma mensagem ao cliente.
        \item \textbf{SUBACK} – é o pacote enviado pelo servidor ao cliente com o objetivo de confirmar o recebimento e processamento de um pacote do tipo SUBSCRIBE. Entre as informações contidas pelo pacote tem uma lista indicando a qualidade do serviço concedida ao cliente para cada tópico e em caso de ter sido negada a subscrição é enviado um sinal justificando o motivo do mesmo.
            UNSUBSCRIBE
    \end{itemize}
\end{frame}

\begin{frame}
    \frametitle{Titulo}
\end{frame}

\section{Características técnicas}\label{Características técnicas}

\section{Segurança}\label{Segurança}

\section{Prática}\label{Prática}
\begin{frame}
    \frametitle{Dispositivos e softwares da parte prática}
    \begin{itemize}
        \item Esp32:
            \begin{itemize}
                \item Sensor Adafruit BMP-280;
                \item Led embutido no Esp32;
                \item C/C++ (Framework Arduino e ESP-IDF);
            \end{itemize}
        \item Raspberry:
            \begin{itemize}
                \item Docker executando o broker mosquitto;
                \item Cliente inscrito no tópico ``\#''
            \end{itemize}
    \end{itemize}
\end{frame}

\begin{frame}
    \frametitle{Diagrama da aplicação}
    \begin{figure}[!htb]
        \centering
        \includegraphics[width=.7\textwidth]{aplication}
        \caption{\label{fig:aplication} Dispositivos e tópicos utilizados}
    \end{figure}
\end{frame}

\begin{frame}[t,fragile,allowframebreaks]{\insertsectionhead}
    \frametitle{Códigos Fonte}
    \begin{lstlisting}
void setup() {
    if (!sensor.begin(BMP280_ADDRESS)) {
        if(!sensor.begin(BMP280_ADDRESS_ALT)) {
            delay(1000);
            ESP.restart();
        }
    }
    pinMode(LED_PIN, OUTPUT);
    wifiConnect();
    MqttConnect();
    xTaskCreate(taskSendTemperature, "send", 20000, NULL, 1, &handle);
}

void loop() {
    if (!mqttClient.connected()) {
        MqttConnect();
    }
    mqttClient.loop();
}
    \end{lstlisting}
\end{frame}

\begin{frame}
    \frametitle{Mão na massa!!}
    \begin{figure}
        \centering
        \includegraphics[width=.3\textwidth]{pizza.png}
        \includegraphics[width=.3\textwidth]{rolo.png}
        \includegraphics[width=.3\textwidth]{burrito.png}
    \end{figure}
\end{frame}

\section{Conclusão}
\begin{frame}
    \frametitle{Agradecimentos}
    \centering
    \Huge{Perguntas?}
    \begin{figure}
        \centering
        \includegraphics[width=.3\textwidth]{alerta.png}
        \includegraphics[width=.3\textwidth]{perigo.png}
        \includegraphics[width=.3\textwidth]{eletricidade.png}
    \end{figure}
    \Huge{Obrigado pela atenção}
\end{frame}

\end{document}
